\Question{Who are the Fit and Proper Judges in the Trial of Witches?}

             The question is whether witches, together with their patrons and protectors and
       defenders, are so entirely subject to the jurisdiction of the Diocesan Ecclesiastical Court and
       the Civil Court so that the Inquisitors of the crime of heresy can be altogether relieved from
       the duty of sitting in judgement upon them. And it is argued that this is so. For the Canon (c.
       accusatus, § sane, lib. VI) says: Certainly those whose high privilege it is to judge
       concerning matters of the faith ought not to be distracted by other business; and Inquisitors
       deputed by the Apostolic See to inquire into the pest of heresy should manifestly not have to
       concern themselves with diviners and soothsayers, unless these are also heretics, nor should
       it be their business to punish such, but they may leave them to be punished by their own
       judges.
             Nor does there seem any difficulty in the fact that the heresy of witches is not
       mentioned in that Canon. For these are subject to the same punishment as the others in the
       court of conscience, as the Canon goes on to say (dist. I, pro dilectione). If the sin of
       diviners and witches is secret, a penance of forty days shall be imposed upon them: if it is
       notorious, they shall be refused the Eucharist. And those whose punishment is identical
       should receive it from the same Court. Then, again, the guilt of both being the same, since
       just as soothsayers obtain their results by curious means, so do witches look for and obtain
       from the devil the injuries which they do to creatures, unlawfully seeking from His creatures
       that which should be sought from God alone; therefore both are guilty of the sin of idolatry.
             This is the sense of Ezechiel xxi, 23; that the King of Babylon stood at the cross-roads,
       shuffling his arrows and interrogating idols.
             Again it may be said that, when the Canon says “Unless these are also heretics,” it
       allows that some diviners and soothsayers are heretics, and should therefore be subject to
       trial by the Inquisitors; but in that case artificial diviners would also be so subject, and no
       written authority for that can be found.
             Again, if witches are to be tried by the Inquisitors, it must be for the crime of heresy;
       but it is clear that the deeds of witches can be committed without any heresy. For when they
       stamp into the mud of the Body of Christ, although this is a most horrible crime, yet it may
       be done without any error in the understanding, and therefore without heresy. For it is
       entirely possible for a person to believe that It is the Lord's body, and yet throw It into the
       mud to satisfy the devil, and this by reason of some pact with him, that he may obtain some
       desired end, such as the finding of a treasure or anything of that sort. Therefore the deeds of
       witches need involved no error in faith, however great the sin may be; in which case they
       are not liable to the Court of the Inquisition, but are left to their own judges.
             Again, Solomon showed reverence to the gods of his wives out of complaisance, and
       was not on that account guilty of apostasy from the Faith; for in his heart he was faithful and
       kept the true Faith. So also when witches give homage to devils by reason of the pact they
       have entered into, but keep the Faith in their hearts, they are not on that account to be
       reckoned as heretics.
             But it may be said that all witches have to deny the Faith, and therefore must be judged
       heretics. On the contrary, even if they were to deny the Faith in their hearts and minds, still
       they could not be reckoned as heretics, but as apostates. But a heretic is different from an
       apostate, and it is heretics who are subject to the Court of the Inquisition; therefore witches
       are not so subject.
             Again it is said, in c. 26, quest. 5: Let the Bishops and their representatives strive by
       every means to rid their parishes entirely of the pernicious art of soothsaying and magic
       derived from Zoroaster; and if they find any man or woman addicted to this crime, let him
       be shamefully cast out of their parishes in disgrace. So when it says at the end of c. 348, Let
       them leave them to their own Judges; and since it speaks in the plural, both of the
       Ecclesiastic and the Civil Court; therefore, according to this Canon they are subject to no
       more than the Diocesan Court.
             But if, just as these arguments seem to show it to be reasonable in the case of
       Inquisitors, the Diocesans also wish to be relieved of this responsibility, and to leave the
       punishment of witches to the secular Courts, such a claim could be made good by the
       following arguments. For the Canon says, c. ut inquisitionis: We strictly forbid the temporal
       lords and rulers and their officers in any way to try to judge this crime, since it is purely an
       ecclesiastical matter: and it speaks of the crime of heresy. It follows therefore that, when the
       crime is not purely ecclesiastical, as is the case with witches because of the temporal injuries
       which they commit, it must be punished by the Civil and not by the Ecclesiastical Court.
             Besides, in the last Canon Law concerning Jews it says: His goods are to be
       confiscated, and he is to be condemned to death, because with perverse doctrine he opposed
       the Faith of Christ. But if it is said that this law refers to Jews who have been converted, and
       have afterwards returned to the worship of the Jews, this is not a valid objection. Rather is
       the argument strengthened by it; because the civil Judge has to punish such Jews as
       apostates from the Faith; and therefore witches who abjure the Faith ought to be treated in
       the same way; for abjuration of the Faith, either wholly or in part, is the essential principle
       of witches.
             And although it says that apostasy and heresy are to be judged in the same way, yet it is
       not the part of the ecclesiastical but of the civil Judge to concern himself with witches. For
       no one must cause a commotion among the people by reason of a trial for heresy; but the
       Governor himself must make provision for such cases.
             The Authentics of Justinian, speaking of ruling princes, says: You shall not permit
       anyone to stir up your Province by reason of a judicial inquiry into matters concerning
       religions or heresies, or in any way allow an injunction to be put upon the Province over
       which you govern; but you shall yourself provide, making use of such monies and other
       means of investigation as are competent, and not allow anything to be done in matters of
       religion except in accordance with our precepts. It is clear from this that no one must meddle
       with a rebellion against the Faith except the Governor himself.
             Besides, if the trial and punishment of such witches were not entirely a matter for the
       civil Judge, what would be the purpose of the laws which provide as follows? All those who
       are commonly called witches are to be condemned to death. And again: Those who harm
       innocent lives by magic arts are to be thrown to the beasts. Again, it is laid down that thy are
       to be subjected to questions and tortures; and that none of the faithful are to associate with
       them, under pain of exile and the confiscation of all their goods. And many other penalties
       are added, which anyone may read in those laws.
              But in contradiction of all these arguments, the truth of the matter is that such witches
       may be tried and punished conjointly by the Civil and the Ecclesiastical Courts. For a
       canonical crime must be tried by the Governor and the Metropolitan of the Province; not by
       the Metropolitan alone, but together with the Governor. This is clear in the Authentics,
       where ruling princes are enjoined as follows: If it is a canonical matter which is to be tried,
       you shall inquire into it together with the Metropolitan of the Province. And to remove all
       doubt on this subject, the gloss says: If it is a simple matter of the observance of the faith,
       the Governor alone may try it; but if the matter is more complicated, then it must be tried by
       a Bishop and the Governor; and the matter must be kept within decent limits by someone
       who has found favour with God, who shall protect the orthodox faith, and impose suitable
       indemnities of money, and keep our subjects inviolate, that is, shall not corrupt the faith in
       them.
              And again, although a secular prince may impose the capital sentence, yet this does not
       exclude the judgement of the Church, whose part it is to try and judge the case. Indeed this
       is perfectly clear from the Canon Law in the chapters de summa trin. and fid. cath., and
       again in the Law concerning heresy, c. ad abolendam and c. urgentis and c.
       excommunicamus, 1 and 2. For the same penalties are provided by both the Civil and the
       Canon Laws, as is shown by the Canon Laws concerning the Manichaean and Arian
       heresies. Therefore the punishment of witches belongs to both Courts together, and not to
       one separately.
              Again, the laws decree that clerics shall be corrected by their own Judges, and not by
       the temporal or secular Courts, because their crimes are considered to be purely
       ecclesiastical. But the crime of witches is partly civil and partly ecclesiastical, because they
       commit temporal harm and violate the faith; therefore it belongs to the Judges of both
       Courts to try, sentence, and punish them.
              This opinion is substantiated by the Authentics, where it is said: If it is an ecclesiastical
       crime needing ecclesiastical punishment and fine, it shall be tried by a Bishop who stands in
       favour with God, and not even the most illustrious Judges of the Province shall have a hand
       in it. And we do not wish the civil Judges to have any knowledge of such proceedings; for
       such matters must be examined ecclesiastically and the souls of the offenders must be
       corrected by ecclesiastical penalties, according to the sacred and divine rules which our laws
       worthily follow. So it is said. Therefore it follows that on the other hand a crime which is of
       a mixed nature must be tried and punished by both courts.
              We make our answer to all the above as follows. Our main object here is to show how,
       with God's pleasure, we Inquisitors of Upper Germany may be relieved of the duty of trying
       witches, and leave them to be punished by their own provincial Judges; and this because of
       the arduousness of the work: provided always that such a course shall in no way endanger
       the preservation of the faith and the salvation of souls. And therefore we engaged upon this
       work, that we might leave to the Judges themselves the methods of trying, judging and
       sentencing in such cases.
              Therefore in order to show that the Bishops can in many cases proceed against witches
       without the Inquisitors; although they cannot so proceed without the temporal and civil
       Judges in cases involving capital punishment; it is expedient that we set down the opinions
       of certain other Inquisitors in parts of Spain, and (saving always the reverence due to them),
       since we all belong to one and the same Order of Preachers, to refute them, so that each
       detail may be more clearly understood.
              Their opinion is, then, that all witches, diviners, necromancers, and in short all who
       practise any kind of divination, if they have once embraced and professed the Holy Faith,
       are liable to the Inquisitorial Court, as in the three cases noted in the beginning of the
       chapter, Multorum querela, in the decretals of Pope Clement concerning heresy; in which it
       says that neither must the Inquisitor proceed without the Bishop, nor the Bishop without the
       Inquisitor: although there are five other cases in which one may proceed without the other,
       as anyone who reads the chapter may see. But in one case it is definitively stated that one
       must not proceed without the other, and that is when the above diviners are to be considered
       as heretics.
              In the same category they place blasphemers, and those who in any way invoke devils,
       and those who are excommunicated and have contumaciously remained under the ban of
       excommunication for a whole year, either because of some matter concerning faith or, in
       certain circumstances, not on account of the faith; and they further include several other
       such offences. And by reason of this the authority of the Ordinary is weakened, since so
       many more burdens are placed upon us Inquisitors which we cannot safely bear in the sight
       of the terrible Judge who will demand from us a strict account of the duties imposed upon
       us.
              And because their opinion cannot be refuted unless the fundamental thesis upon which
       it is founded is proved unsound, it is to be noted that it is based upon the commentators on
       the Canon, especially on the chapter accusatus, and § sane, and on the words “savour of
       heresy.” Also they rely upon the sayings of the Theologians, S. Thomas, Blessed Albert, and
       S. Bonaventura, in the Second Book of Sentences, dist. 7.
              It is best to consider some of these in detail. For when the Canon says, as was shown in
       the first argument, that the Inquisitors or heresy should not concern themselves with
       soothsayers and diviners unless they manifestly savour of heresy, they say that soothsayers
       and diviners are of two sorts, either artificial or heretical. And the first sort are called
       diviners pure and simple, since they work merely by art; and such are referred to in the
       chapter de sortilegiis, where it says that the presbyter Udalricus went to a secret place with a
       certain infamous person, that is, a diviner, says the gloss, not with the intention of invoking
       the devil, which would have been heresy, but that, by inspecting the astrolabe, he might find
       out some hidden thing. And this, they say, is pure divination or sortilege.



	   %note (begin)
           “Arrows.” Esarhaddon is employing a mode of sortilege by arrows, belomancy, which
        was extensively practised among the Chaldeans, as also among the Arabs. Upon this text S.
        Jerome comments: “He shall stand in the highway, and consult the oracle after the manner
        of his nation, that he may cast arrows into a quiver, and mix them together, being written
        upon or marked with the names of each people, that he may see whose arrow will come
        forth, and which city he ought first to attack.” The arrows employed by the Arabs were often
        three in number, upon the first of which was inscribed, “My Lord hath commanded me”;
        upon the second, “My Lord hath forbidden me”; and the third was blank. If the inquirer
        drew the first it was an augury of success; the second gave an omen of failure; if the third
        were drawn, all three were mixed again and another trial was made. In some countries
        diving rods were employed instead of arrows. These were drawn from a vessel, or, it might
        be, cast into the air, the position in which they fell being carefully noted. This practice is
        rhabdomancy. The LXX, “Ezechiel” xxi, 21, reads                         , not                  ,
        and rhabdomancy is mentioned by S. Cyril of Alexandria. The “Koran,” V, forbids
        prognostication by divining arrows, which are there denounced as “an abomination of the
        work of Satan.” See my “History of Witchcraft,” Chap. V, pp. 182-83.



          “Manichaean.” For the close connexion between the Manichees and witches see my
        “History of Witchcraft,” Chap. I.



           “Pope Clement.” Pope Clement V, born at Villandraut, 1264; elected to the Chair of S.
        Peter, 5 June, 1305; died at Roquemare, 20 April, 1314; completed the mediaeval “Corpus
        Iuris Canonici” by the publication of a collection of papal decretals known as
        “Clementinae” or “Liber Clementinarum,” sometimes as “Liber Septimus” in reference to
        “Liber Sextus” of Bonafice VIII. It contains decretals of this latter Pontiff, of Benedict XI,
        and of Clement himself. Together with the decrees of the Council of Vienne it was
        promulgated, 12 March, 1314, at the Papal residence of Monteaux near Carpentras. It is
        divided into five books with subdivisions of titles and chapters. As Clement V died before the
        collection had been generally published, John XXII promulgated it anew, 25 October, 1317,
        and sent it to the University of Bologna as the authorative Corpus of decretals to be used in
        the courts and schools.
		%note (end)

             But the second sort are called heretical diviners, whose art involves some worship of or
       subjection to devils, and who essay by divination to predict the future of something of that
       nature, which manifestly savours of heresy; and such are, like other heretics, liable to the
       Inquisitorial Court.
             And that this is the meaning of the Canon they prove from commentaries of the
       Canonists on the word “savour.” For Giovanni d’Andrea, writing on this Canon accusatus,
       and the word “saviour,” says: They savour of heresy in this way, that they utter nefarious
       prayers and offer sacrifices at the altars of idols, and they consult with devils and receive
       answers from them; or they meet together to practise heretical sortes, that they may have an
       answer, re-baptize a child, and practise other such matters.
             Many others also they quote in support of their opinion, including John Modestus; S.
       Raymund, and William de Laudun, O.P. And they refer to the decision of the Church at the
       Council of Aquitaine, c. 26, q. 5, Episcopi, where such superstitious women are called
       infidels, saying, Would that these had perished alone in their perfidy. And perfidy in a
       Christian is called heresy; therefore they are subject to the Court of the Inquisitors of heresy.
             They quote also the Theologians, especially S. Thomas, the Second Book of Sentences,
       dist. 7, where he considers whether it is a sin to use the help of devils. For speaking of that
       passage in Esaias viii: Should not a people seek unto their God? he says among other things:
       In everything the fulfilment of which is looked for from the power of the devil, because of a
       pact entered into with him, there is apostasy from the faith, either in word, if there is some
       invocation, or in deed, even if there be no sacrifice offered.
             To the same effect they quote Albertus, and Peter of Tarentaise, and Giovanni
       Bonaventura, who has lately been canonized, not under the name of Giovanni, although that
       was his true name. Also they quote Alexander of Hales and Guido the Carmelite. All these
       say that those who invoke devils are apostates, and consequently heretics, and therefore
       subject to the Court of the Inquisitors of heretics.
             But the said Inquisitors of Spain have not, by the above or any other arguments, made
       out a sufficient case to prove that such soothsayers etc. may not be tried by the Ordinary or
       the Bishops without the Inquisitors; and that the Inquisitors may not be relieved from the
       duty of trying such diviners and necromancers, and even witches: not that the Inquisitors are
       not rather to be praised than blamed when they do try such cases, when the Bishops fail to
       do so. And this is the reason that they have not proved their case. The Inquisitors need only
       concern themselves with matters of heresy, and the heresy must be manifest; as is shown by
       the frequently quoted Canon accusatus, § sane.
             This being the case, it follows that however serious and grave may be the sin which a
       person commits, if it does not necessarily imply heresy, then he must not be judged as a
       heretic, although he is to be punished. Consequently an Inquisitor need not interfere in the
       case of a man who is to be punished as a malefactor, but not as a heretic, but may leave him
       to be tried by the Judges of his own Province.
             It follows again that all the crimes of invoking devils and sacrificing to them, of which
       the Commentators and Canonists and Theologians speak, are no concern of the Inquisitors,
       but can be left to the secular or episcopal Courts, unless they also imply heresy. This being
       so, and it being the case that the crimes we are considering are very often committed without
       any heresy, those who are guilty of such crimes are not to be judged or condemned as
       heretics, as is proved by the following authorities and arguments.
             For a person rightly to be adjudged a heretic he must fulfil five conditions. First, there
       must be an error in his reasoning. Secondly, that error must be in matters concerning the
       faith, either being contrary to the teaching of the Church as to the true faith, or against sound
       morality and therefore not leading to the attainment of eternal life. Thirdly, the error must lie
       in one who has professed the Catholic faith, for otherwise he would be a Jew or a Pagan, not
       a heretic. Fourthly, the error must be of such a nature that he who holds it must confess
       some of the truth of Christ as touching either His Godhead or His Manhood; for is a man
       wholly denies the faith, he is an apostate. Fifthly, he must pertinaciously and obstinately
       hold to and follow that error. And that this is the sense of heretics is proved as follows (not
       by way of refuting, but of substantiating the gloss of the Canonists).
             For it is well known to all through common practice that the first essential of a heretic
       is an error in the understanding; but two conditions are necessary before a man can be called
       a heretic; the first material, that is, an error in reasoning, and the second formal, that is, an
       obstinate mind. S. Augustine shows this when he says: A heretic is one who either initiates
       or follows new and false opinions. It can also be proved by the following reasoning: heresy
       is a form of infidelity, and infidelity exists subjectively in the intellect, in such a way that a
       man believes something which is quite contrary to the true faith.
             This being so, whatever crime a man commits, if he acts without an error in his
       understanding he is not a heretic. For example, if a man commits fornication or adultery,
       although he is disobeying the command Thou shalt not commit adultery, yet he is not a
       heretic unless he holds the opinion that it is lawful to commit adultery. The point can be
       argued in this way: When the nature of a thing is such that two constituent parts are
       necessary to its existence, if one of those two parts is wanting the thing itself cannot exist;
       for if it could, then it would not be true that that part is necessary to its existence. For in the
       constitution of a house it is necessary that there should be a foundation, walls, and a roof;
       and if one of these is missing, there is no house. Similarly, since an error in the
       understanding is a necessary condition of heresy, no action which is done entirely without
       any such error can make a man a heretic.
             Therefore we Inquisitors of Germany are in agreement with Blessed Antoninus where
       he treats of this matter in the second part of his Summa; saying that to baptize things, to
       worship devils, to sacrifice to them, to tread underfoot the Body of Christ, and all such
       terrible crimes, do not make a man a heretic unless there is an error in his understanding.
       Therefore a man is not a heretic who, for example, baptizes an image, not holding any
       erroneous belief about the Sacrament of Baptism or its effect, nor thinking that the baptism
       of the image can have any effect of its own virtue; but does this in order that he may more
       easily obtain some desire from the devil whom he seeks to please by this means, acting with
       either an implied or an expressed pact that the devil will fulfil the desires either of himself or
       of someone else. In this way men who, with either a tacit or an expressed pact, invoke devils
       with characters and figures in accordance with magic practice to perform their desires are
       not necessarily heretics. But they must not ask from the devil anything which is beyond the
       power or the knowledge of the devil, having a wrong understanding of his power and
       knowledge. Such would be the case with any who believed that the devil could coerce a
       man's free will; or that, by reason of their pact with him, the devil could do anything which
       they desired, however much it were forbidden by God; or that the devil can know the whole
       of the future; or that he can effect anything which only God can do. For there is no doubt
       that men with such beliefs have an error in their understanding, holding a wrong opinion of
       the power of the devil; and therefore, granting the other conditions necessary for heresy,
       they would be heretics, and would be subject at once to the Ordinary and to the Inquisitorial
       Court.
             But if they act for the reasons we have said, not out of any wrong belief concerning
       baptism or the other matters we have mentioned, as they very commonly do; for since
       witches and necromancers know that the devil is the enemy of the faith and the adversary of
       salvation, it must follow that they are compelled to believe in their hearts that there is great
       might in the faith and that there is no false doctrine of which the father of lies is not known
       to be the origin; then, although they sin most grievously, yet they are not heretics. And the
       reason is that they have no wrong belief concerning the sacrament, although they use it
       wrongly and sacrilegiously. Therefore they are rather sorcerers than heretics, and are to be
       classed with those whom the above Canon accusatus declares are not properly subject to the
       Inquisitorial Court, since they do not manifestly savour of heresy; their heresy being hidden,
       if indeed it exists at all.
             It is the same with those who worship and sacrifice to the devil. For if they do this in
       the belief that there is any divinity in devils, or that they ought to be worshipped and that, by
       reason of such worship, that can obtain from the devil what they desire in spite of the
       prohibition or permission of God, then they are heretics. But if they act in such a way not
       out of any such belief concerning the devil, but so that they may the more easily obtain their
       desires because of some pact formed with the devil, then they are not necessarily heretics,
       although they sin most grievously.
             For greater clearness, some objections are to be disposed of and refuted. For it appears
       to be against our argument that, according to the laws, a simonist is not a heretic (1, q. 1:
       “Whoever by means of money, but not having an error of the understanding”). For a
       simonist is not in the narrow and exact sense of the word a heretic; but broadly speaking and
       by comparison he is so, according to S. Thomas, when he buys or sells holy things in the
       belief that the gift of grace can be had for money. But if, as is often the case, he does not act
       in this belief, he is not a heretic. Yet he truly would be if he did believe that the gift of grace
       could be had for money.
             Again we are apparently in opposition to what is said of heretics in the Canon; namely,
       that he who reveres a heretic is himself a heretic, but he who worships the devil sins more
       heavily than he who reveres a heretic, therefore, etc.
             Also, a man must be obviously a heretic in order that he may be judged as such. For the
       Church is competent to judge only of those things which are obvious, God alone having
       knowledge and being the Judge of that which is hidden (dist. 33, erubescant). But the inner
       understanding can only be made apparent by intrinsic actions, either seen or proved;
       therefore a man who commits such actions as we are considering is to be judged a heretic.
             Also, it seems impossible that anyone should commit such an action as the treading
       underfoot of the Body of Christ unless he held a wrong opinion concerning the Body of
       Christ; for it is impossible for evil to exist in the will unless there is error in the
       understanding. For according to Aristotle every wicked man is either ignorant or in error.
       Therefore, since they who do such things have evil in their wills, they must have an error in
       their understandings.
             To these three objections we answer as follows; and the first and third may be
       considered together. There are two kinds of judgement, that of God and that of men. God
       judges the inner man; whereas man can only judge of the inner thoughts as they are reflected
       by outer actions, as is admitted in the third of these arguments. Now he who is a heretic in
       the judgement of God is truly and actually a heretic; for God judges no one as a heretic
       unless he has some wrong belief concerning the faith in his understanding. But when a man
       is a heretic in the judgement of men, he need not necessarily be actually a heretic; but
       because his deeds give an appearance of a wrong understanding of the faith he is, by legal
       presumption, considered to be a heretic.
             And if it be asked whether the Church should stigmatize at once as heretics those who
       worship devils or baptize imagines, note these answers. First, it belongs rather to the
       Canonists than to the Theologians to discriminate in this matter. The Canonists will say that
       they are by legal presumption to be considered as heretics, and to be punished as such. A
       Theologian will say that it is in the first instance a matter for the Apostolic See to judge
       whether a heresy actually exists or is only to be presumed in law. And this may be because
       whenever an effect can proceed from a twofold cause, no precise judgement can be formed
       of he actual nature of the cause merely on the basis of the effect.
             Therefore, since such effects as the worship of the devil or asking his help in the
       working of witchcraft, by baptizing an image, or offering to him a living child, or killing an
       infant, and other matters of this sort, can proceed from two separate causes, namely, a belief
       that it is right to worship the devil and sacrifice to him, and that images can receive
       sacraments; or because a man has formed some pact with the devil, so that he may obtain
       the more easily from the devil that which he desires in those matters which are not beyond
       the capacity of the devil, as we have explained above; it follows that no one ought hastily to
       form a definite judgement merely on the basis of the effect as to what is its cause, that is,
       whether a man does such things out of a wrong opinion concerning the faith. So when there
       is no doubt about the effect, still it is necessary to inquire farther into the cause; and if it be
       found that a man has acted out of a perverse and erroneous opinion concerning the faith,
       then he is to be judged a heretic and will be subject to trial by the Inquisitors together with
       the Ordinary. But if he has not acted for these reasons, he is to be considered a sorcerer, and
       a very vile sinner.


	   %note (begin)
           “Bonaventura.” The parents of S. Bonaventura were Giovanni di Fidanze and Maria
        Ritella. He was born at Bagnorea, near Viterbo, in 1221, and baptized Giovanni. This was
        changed to Bonaventura owing to the exclamation of S. Francis, “O buona ventura,” when
        the child was brought to him to be cured of a dangerous illness. (This account has been
        doubted, and it is true that others bore the name before S. Bonaventura.) S. Bonaventura
        was canonized by Sixtus IV, 14 April, 1482. This formal enrolment in the catalogue of the
        Saints was thus long delayed mainly owing to the unfortunate dissensions concerning
        Franciscan affairs after the Saint's death, 15 July, 1274. He was inscribed among the
        principal Doctors of the Church by Sixtus V, 14 March, 1587. His feast is celebrated 14
        July.



           “Guido the Carmelite.” Guy de Perpignan, “Doctor PArisiensis,” d. 1342; General of
        the Carmelite Order from 1318-20. His chief work was the “Summa de Haeresibus.”
		%note (end)

             Another answer which touches the matter nearly is that, whatever may be said and
       alleged, it is agreed that all diviners and witches are judged as heretics by legal presumption
       and not by actual fact are subject to the Court of the Ordinary, not of the Inquisitors. And
       the aforesaid Inquisitors of other countries cannot defend their opinions by quoting the
       Canon and its commentators, because they who sacrifice to and worship devils are judged to
       be heretics be legal presumption, and not because the facts obviously show that they are
       such. For the text says that they must savour of heresy manifestly, that is, intrinsically and
       by their very nature. And it is enough for us Inquisitors to concern ourselves with those who
       are manifestly from the instrinsic nature of the case heretics, leaving others to their own
       judges.
             It has been said that the cause must be inquired into, to know whether or not a man acts
       out of an error of faith; and this is easy. For the spirit of faith is known by the act of faith; as
       the spirit of chastity is shown by a chaste life; similarly the Church must judge a man a
       heretic if his actions show that he disputes any article of the faith. In this way even a witch,
       who has wholly or in part denied the faith, or used vilely the Body of Christ, and offered
       homage to the devil, may have done this merely to propitiate the devil; and even if she has
       totally denied the faith in her heart, she is to be judged as an apostate, for the fourth
       condition, which is necessary before anyone can rightly be said to be a heretic, will be
       wanting.
             But if against this conclusion be set the Bull and commission given to us by our Holy
       Father Innocent VIII, that witches should be tried by the Inquisitors, we answer in this way.
       That this is not to say that the Diocesans also cannot proceed to a definite sentence against
       witches, in accordance with those ancient laws, as has been said. For that Bull was rather
       given to us because of the great care with which we have wrought to the utmost of our
       ability with the help of God.
             Therefore we cannot concede to those other Inquisitors their first argument, since the
       contrary conclusion is rather the true one; for simonists are thought to be heretics only be
       legal presumption, and the Ordinaries themselves without the Inquisitors can try them.
       Indeed, the Inquisitors have no need to concern themselves with various simonists, or
       similarly with any others who are judged to be heretics only by legal presumption. For they
       cannot proceed against schismatic Bishops and other high Dignitaries, as is shown by the
       chapter of the Inquisition Concerning Heretics, Book VI, where is says: The Inquisitors of
       the sin of heresy deputed by the Apostolic See or by any other authority have no power to
       try such offenders on this sort of charge, or to proceed against them under pretext of their
       office, unless it is expressly stated in the letters of commission from the Apostolic See that
       they are empowered to do so.
             But if the Inquisitors know or discover that Bishops or other high Dignitaries have been
       charged with heresy, or have been denounced or suspected of that crime, it is their duty to
       report the fact to the Apostolic See.
             Similarly the answer to their second argument is clear from what has been said. For he
       who cherishes and comforts a heretic is himself a heretic if he does this in the belief that he
       is worthy to be cherished or honoured on account of his doctrine or opinion. But if he
       honours him for some temporal reason, without any error of faith in his understanding, he is
       not rightly speaking a heretic, though he is so by a legal fiction or presumption or
       comparison, because he acts as if he held a wrong belief concerning the faith like him whom
       he cherishes: so in this case he is not subject to the Inquisitorial Court.
             The third argument is similarly answered. For though a man should be judged by the
       Church as a heretic on account of his outward actions, visible and proved, yet it does not
       always follow that he is actually a heretic, but is only so reputed by legal presumption.
       Therefore in this case he is not liable to be tried by the Inquisitorial Court, because he does
       not manifestly savour of heresy.
             For their fourth argument, it is a false assumption to say that it is not possible for
       anyone to tread underfoot the Body of Christ unless he has some perverse and wrong belief
       concerning the Body of Christ. For a man may do this with a full knowledge of his sin, and
       with a firm belief that the Body of Christ is truly there. But he does it to please the devil,
       and that he may more easily obtain his desire from him. And though in every sin there is an
       error, it need not necessarily be an error of the understanding, which is heresy or a wrong
       belief concerning the faith; for it may be an erroneous use of some power which turns it to
       vicious purposes; and then it will only be the first of those five conditions which are
       necessary constituents of heresy, in accordance with which a heretic is rightly liable to the
       Inquisitorial Court.
             And it is not a valid objection to say that an Inquisitor may, nevertheless, proceed
       against those who are denounced as heretics, or are under a light or a strong or a grave
       suspicion of heresy, although they do not appear to savour manifestly of heresy. For we
       answer that an Inquisitor may proceed against such in so far as they are denounced or
       suspected for heresy rightly so called; and this is the sort of heresy of which we are speaking
       (as we have often said), in which there is an error in the understanding, and the other four
       conditions are superadded. And the second of these conditions is that such error should
       consist in matters concerning the faith, or should be contrary to the true decisions of the
       Church in matters of faith and good behaviour and that which is necessary for the attainment
       of eternal life. For if the error be in some matter which does not concern the faith, as, for
       example, a belief that the sun is not greater than the earth, or something of that sort, then it
       is not a dangerous error. But an error against Holy Scripture, against the articles of the faith,
       or against the decision of the Church, as has been said above, is heresy (art. 24, q. 1, haec
       est fides).
             Again, the determination of doubts respecting the faith belongs chiefly to the Church,
       and especially to the Supreme Pontiff, Christ's Vicar, the successor of S. Peter, as is
       expressly stated (art. 24, q. 1, quotiens). And against the determination of the Church, as S.
       Thomas says, art. 2, q. 2, no Doctor or Saint maintains his own opinion; not S. Jerome nor S.
       Augustine nor any other. For just as he who obstinately argues against the faith is a heretic,
       so also is he who stubbornly maintains his opinion against the determination of the Church
       in matters concerning the faith and that which is necessary for salvation. For the Church
       herself has never been proved to be in error over matter of faith (as it is said in art. 24, q. 1,
       a recta, and in other chapters). And it is expressly said, that he who maintains anything
       against the determination of the Church, not in an open and honest manner, but in matters
       which concern faith and salvation, is a heretic. For he need not be a heretic because he
       disagrees over other matters, such as the separability of law from use in matters which are
       affected by use: this matter has been settled by Pope John XXII in his Extrauagantes, where
       he says that they who contradict this opinion are stubborn and rebellious against the Church,
       but not heretics.
             The third condition required is that he who holds the error should be one who has
       professed the Catholic faith. For is a man has never professed the Christian faith, he is not a
       heretic but simply an infidel, like the Jews or the Gentiles who are outside the faith.
       Therefore S. Augustine says in the City of God: The devil, seeing the human race to be
       delivered from the worship of idols and devils, stirred up heretics who, under the guise of
       Christians, should oppose Christian doctrine. So for a man to be a heretic it is necessary that
       he should have received the Christian faith in baptism.
             Fourthly, it is necessary that the man who so errs should retain some of the true belief
       concerning Christ, pertaining either to His divinity or to His humanity. For if he retains no
       part of the faith, he is more rightly to be considered an apostate than a heretic. In this way
       Julian was an apostate. For the two are quite different, though sometimes they are confused.
       For in this manner there are found to be men who, driven by poverty and various afflictions,
       surrender themselves body and soul to the devil, and deny the faith, on condition that the
       devil will help them in their need to the attainment of riches and honours.
             For we Inquisitors have known some, of whom a few afterwards repented, who have
       behaved in this way merely for the sake of temporal gain, and not through any error in the
       understanding; wherefore they are not rightly heretics, nor even apostates in their hearts, as
       was Julian, though they must be reckoned as apostates.
             They who are apostates in their heart and refuse to return to the faith are, like
       impenitent heretics, to be delivered to the secular Court. But if they are desirous of
       reconciliation, they are received back into the Church, like penitent heretics. See the chapter
       ad abolendam, § praesenti, de haeretic., lib. 6. Of the same opinion is S. Raymund in his
       work de Apostolica, cap. reuertentes, where he says that they who return from the perfidy of
       apostasy, though they were heretics, are to be received back like penitent heretics. And here
       the two are confused, as we have said. And he adds: Those who deny the faith through fear
       of death (that is, who deny the faith for the sake of temporal gain from the devil, but do not
       believe their error) are heretics in the sight of the law, but are not, properly speaking,
       heretics. And he adds: Although they have no erroneous belief, yet since the Church must
       judge by outward signs they are to be considered as heretics (not this fiction of law); and if
       they return, they are to be received as penitent heretics. For the fear of death, or the desire
       for temporal gain, is not sufficient to cause a constant man to deny the faith of Christ.
       Wherefore he concludes that it is more holy to die than to deny the faith or to be fed by
       idolatrous means, as S. Augustine says.
             The judgement of witches who deny the faith would be the same; that when they wish
       to return they should be received as penitents, but otherwise they should be left to the
       secular Court. But they are by all means to be received back into the bosom of the Church
       when they repent; and are left to the secular Court if they will not return; and this is because
       of the temporal injuries which they cause, as will be shown in the methods of passing
       sentence. And all this may be done by the Ordinary, so that the Inquisitor can leave his
       duties to him, at least in a case of apostasy; for it is otherwise in other cases of sorcerers.
             The fifth condition necessary for a man to be rightly thought a heretic is that he should
       obstinately and stubbornly persist in his error. Hence, according to S. Jerome, the
       etymological meaning of heresy is Choice. And again S. Augustine says: Not he who
       initiates or follows false doctrines, but he who obstinately defends them, is to be considered
       a heretic. Therefore if anyone does not evilly persist in believing some false doctrine, but
       errs through ignorance and is prepared to be corrected and to be shown that his opinion is
       false and contrary to Holy Scripture and the determination of the Church, he is not a heretic.
       For he was ready to be corrected when his error was pointed out to him. And it is agreed that
       every day the Doctors have various opinions concerning Divine matters, and sometimes they
       are contradictory, so that one of them must be false; and yet none of them are reputed to be
       false until the Church has come to a decision concerning them. See art. 24, q. 3, qui in
       ecclesia.
             From all this is is concluded that the sayings of the Canonists on the words “savour
       manifestly of heresy” in the chapter accusatus do not sufficiently prove that witches and
       others who in any way invoke devils are subject to trial by the Inquisitorial Court; for it is
       only by a legal fiction that they judge such to be heretics. Neither is it proved by the words
       of the Theologians; for they call such persons apostates either in word or in deed, but not in
       their thoughts and their hearts; and it is of this last error that the words “savour of heresy”
       speak.
             And though such persons should be judged to be heretics, it does not follow form this
       that a Bishop cannot proceed against them without an Inquisitor to a definite sentence, or
       punish them with imprisonment or torture. More than this, even when this decision does not
       seem enough to warrant the exemption of us Inquisitors from the duty of trying witches, still
       we are unwilling to consider that we are legally compelled to perform such duties ourselves,
       since we can depute the Diocesans to our office, at least in respect of arriving at a
       judgement.


	   %note (begin)
            “Extrauagantes.” This word designates some Papal decretals not contained in certain
        canonical collections which possess a special authority, that is, they are not found in (but
        “wander outside,” “extra uagari”) the Decree of Gratian, or the three great official
        collections of the “Corpus Iuri” (the Decretals of Gregory IX; the Sixth Book of the
        Decretals; and the Clementines). The term is now applied to the collections known as the
        “Extrauagantes Ioannis XXII” and the “Extrauagantes Communes.” When John XXII
        (1316-34) published the Decretals already known as “Clementines,” there also existed
        various pontifical documents, obligatory upon the whole Church indeed, but not included in
        the “Corpus Iuris,” and these were called “Extrauagantes.” In 1325 Zenselinus de
        Cassanis added glosses to twenty constitutions of John XXII, and named this collection
        “Uiginti Extrauagantes papae Ioannis XXII.” Chappuis also classified these under fourteen
        titles containing all twenty chapters.
	   %note (end)


             For this provision is made in the Canon Law (c. multorum in prin. de haeret. in Clem.).
       There it says: As a result of a general complaint, and that this sort of Inquisition may
       proceed more fortunately and the inquiry into this crime be conducted more skilfully,
       diligently, and carefully, we order that this kind of case may be tried by the Diocesan
       Bishops as well as by the Inquisitors deputed by the Apostolic See, all carnal hatred or fear
       or any temporal affection of this sort being put aside; and so either of the above may move
       without the other, and arrest or seize a witch, placing her in safe custody in fetters and iron
       chains, if it seems good to him; and in this matter we leave the conduct of the affair to his
       own conscience; but there must be no negligence in inquiring into such matters in a manner
       agreeable to God and justice; but such witches must be thrust into prison rather as a matter
       of punishment than custody, or be exposed to torture, or be sentenced to some punishment.
       And a Bishop can proceed without an Inquisitor, or an Inquisitor without a Bishop; or, if
       either of their offices be vacant, their deputies may act independently of each other,
       provided that is is impossible for them to meet together for joint action within eight days of
       the time when the inquiry is due to commence; but if there be no valid reason for their not
       meeting together, the action shall be null and void in law.
             The chapter proceeds to support our contention as follows: But if the Bishop or the
       Inquisitor, or either of their deputies, are unable or unwilling, for any of the reasons which
       we have mentioned, to meet together personally, they can severally depute their duties to
       each other, or else signify their advice and approval by letters.
             From this it is clear that even in those cases where the Bishop is not entirely
       independent of the Inquisitor, the Inquisitor can depute the Bishop to act in his stead,
       especially in the matter of passing sentence: therefore we ourselves have decided to act
       according to this decision, leaving other Inquisitors to other districts to act as seems good to
       them.
             Therefore in answer to the arguments, it is clear that witches and sorcerers have not
       necessarily to be tried by the Inquisitors. But as for the other arguments which seek to make
       it possible for the Bishops in their turn to be relieved from the trial of witches, and leave this
       to the Civil Court, it is clear that this is not so easy in their case as it is in that of the
       Inquisitors. For the Canon Law (c. ad abolendam, c. uergentis, and c. excommunicamus
       utrumque) says that in a case of heresy it is for the ecclesiastical judge to try and to judge,
       but for the secular judge to carry out the sentence and to punish; that is, when a capital
       punishment is in question, though it is otherwise with other penitential punishments.
             It seems also that in the heresy of witches, though not in the case of other heresies, the
       Diocesans also can hand over to the Civil Courts the duty of trying and judging, and this for
       two reasons: first because, as we have mentioned in our arguments, the crime of witches is
       not purely ecclesiastical, being rather civil on account of the temporal injuries which they
       commit; and also because special laws are provided for dealing with witches.
            Finally, it seems that in this way it is easiest to proceed with the extermination of
       witches, and that the greatest help is thus given to the Ordinary in the sight of that terrible
       Judge who, as the Scriptures testify, will exact the strictest account from and will most
       hardly judge those who have been placed in authority. Accordingly we will proceed on this
       understanding, namely, that the secular Judge can try and judge such cases, himself
       proceeding to the capital punishment, but leaving the imposition of any other penitential
       punishment to the Ordinary.

	   %section
                     A Summary or Classification of the Matters Treated of in this Third Part.

             In order, then, that the Judges both ecclesiastical and civil may have a ready knowledge
       of the methods of trying, judging and sentencing in these cases, we shall proceed under three
       main heads. First, the method of initiating a process concerning matters of the faith; second,
       the method of proceeding with the trial; and third, the method of bringing it to a conclusion
       and passing sentence on witches.
             The first head deals with five difficulties. First, which of the three methods of
       procedure provided by the law is the most suitable. Second, the number of witnesses. Third,
       whether these can be compelled to take the oath. Fourth, the condition of the witnesses.
       Fifth, whether mortal enemies may be allowed to give evidence.
             The second head contained eleven Questions. I. How witnesses are to be examined, and
       that there should always be five persons present. Also how witches are to be interrogated,
       generally and particularly. (This will be numbered the Sixth Question of the whole Part; but
       we alter the numeration here to facilitate reference by the reader). II. Various doubts are
       cleared up as to negative answers, and when a witch is to be imprisoned, and when she is to
       be considered as manifestly guilty of the heresy of witchcraft. III. The method of arresting
       witches. IV. Of two duties which devolve upon the Judge after the arrest, and whether the
       names of the deponents should be made known to the accused. V. Of the conditions under
       which an Advocate shall be allowed to plead for the defence. VI. What measures the
       Advocate shall take when the names of the witnesses are not made known to him, and when
       he wishes to protest to the Judge that the witnesses are mortal enemies of the prisoner. VII.
       How the Judge ought to investigate the suspicion of such mortal enmity. VIII. Of the points
       which the Judge must consider before consigning the prisoner to torture. IX. Of the method
       of sentencing the prisoner to examination by torture. X. Of the method of proceeding with
       the torture, and how they are to be tortured; and of the provisions against silence on the part
       of the witch. XI. Of the final interrogations and precautions to be observed by the Judge.
             The third head contains first of all three Questions dealing with matters which the
       Judge must take into consideration, on which depends the whole method of passing
       sentence. First, whether a prisoner can be convicted by a trial of red-hot iron. Second, of the
       method in which all sentences should be passed. Third, what degrees of suspicion can justify
       a trial, and what sort of sentence ought to be passed in respect of each degree of suspicion.
       Finally, we treat of twenty methods of delivering sentence, thirteen of which are common to
       all kinds of heresy, and the remainder particular to the heresy of witches. But since these
       will appear in their own places, for the sake of brevity they are not detailed here.
